\chapter{Theoretical background}\label{chap_teo}


This chapter presents the theoretical background necessary to work with the World Bank's data presented on chapter \ref{chap_data}. As chapter \ref{chap_data} shows, this project uses three main data sources, the World Bank's world development indicators taken from the Data Bank \parencite{wb_data}, all the Major and Historic Awards given by the world Bank since 1989 to 2014 \parencite{wb_data}, and the private Investigations data provided by the Integrity Vice Presidency Unit (INT). As it will be shown later on chapter \ref{chap_prod}, the data product involves using techniques such data mining, machine learning modeling and data visualization. 
Section  \ref{sec_scrape} explains how to obtain the public data and the data mining tools needed for this task. This chapters includes some basic Natural Language Processing (NLP) techniques that were later used in the name disambiguation (see chapter \ref{chap_prod} for details). Then, section \ref{sec_model_teo} explains the background behind the machine leaning models used to predict corruption collusion and fraud. Finally, section \ref{sec_vis_teo} presents some of the basic visualization techniques used in the web application.

\section{Scraping the data} \label{sec_scrape}

This section explains how to obtain the public data; namely, the World Bank's world development indicators and all the Major and Historic Awards since 1989 to 2014.



\section{Models} \label{sec_model_teo}

\section{Visualization techniques} \label{sec_vis_teo}