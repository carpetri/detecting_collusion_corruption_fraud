% !TEX root = A0_thesis_detecting_collusion_corruption_fraud.tex
\newpage
\thispagestyle{plain}
\pagenumbering{arabic}
\begin{center}
\scshape \large Detecting Corruption Collusion and Fraud\footnote{ Last Update: \today.}
% \scshape \large \thetitle\footnote{  Last Update: \today.}
% Tesis de Licenciatura para obtener el grado de Licenciado en Matemáticas Aplicadas por el Instituto Tecnológico Autónomo de México (ITAM). México D.F., 2015.  \\}
\\

\scshape \theauthor\footnote{ I especially thank Rayid Ghani (Ph.D. Carnegie Mellon University) and  Eric Rozier (Ph.D. University of Illinois Urbana-Champaign) for their mentorship throughout the elaboration of this work. I also appreciate the collaboration and comments of Jeff Alstott (Ph.D. Cambridge), Dylan Fitzpatrick (Ph.D. Carnegie Mellon), Misha Teplitskiy (Ph.D. University of Chicago) and Elizabeth Wiramidjaja (World Bank).
\\  \href{mailto:carpetri@gmail.com}{\texttt{carpetri@gmail.com}}  \hfill \href{http://detecting-corruption.carlospetricioli.com}{\texttt{detecting-corruption.carlospetricioli.com}} }
\normalsize
\end{center}

\section*{\centering  \normalsize{Abstract} }
% \begin{quotation}

\small
\noindent The World Bank Group lends billions of dollars each year to fund development projects in its efforts to reduce global poverty. This project helps investigators at the Bank search for patterns of collusion, corruption, and fraud in its contracts data by using models of contract-specific risk. An automated approach like this can  highly increase the World Bank's ability to detect these offenses by efficiently targeting  their future investigations. 

The actual problem is that contractors providing goods and services on World Bank's projects are typically hired through a competitive bidding process, but occasionally, prospective contractors influence the competitive system by colluding with other contractors, bribing government officials, or otherwise manipulating the bidding process. These offenses have far-reaching effects on the price and quality of contract delivery. This project develops contract-level risk models that use historical data on major contracts awarded from the past 20 years and internal investigations data, covering companies and projects investigated  in the past.

To approach the problem,  an interactive dashboard was developed for investigators to track any company's activity across countries, sectors, and time. By using this tool, investigators can track contract awards companies have received, including under different names, view a risk score for each World Bank contract, as calculated by our contract risk model and visualize the immediate neighborhood of the company in its co-award network.

In conclusion, current data is sufficient to forecast risk and allows  investigators to be proactive in determining which companies, projects and contracts to examine.
% \end{quotation}
\thispagestyle{plain}
\vfill
\scriptsize \noindent \textbf{Keywords:} Machine Learning, Corruption, Fraud, The World Bank.
\normalsize