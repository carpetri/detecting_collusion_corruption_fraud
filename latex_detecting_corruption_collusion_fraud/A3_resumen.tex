% !TEX root = A0_thesis_detecting_collusion_corruption_fraud.tex
\thispagestyle{plain}
\pagenumbering{arabic}
\begin{center}
% \scshape \large Detecting Corruption Collusion and Fraud\footnote{ Last Update: \today.}
\scshape \normalsize Cómo identificar corrupción, colusión y fraude en los contratos que otorga el Banco Mundial
% Tesis de Licenciatura para obtener el grado de Licenciado en Matemáticas Aplicadas por el Instituto Tecnológico Autónomo de México (ITAM). México D.F., 2015.  \\}
\bigskip

\scshape \normalsize \theauthor\footnote{ \tiny Agradezco al Dr. Rayid Ghani  y al  Dr.  Eric Rozier por su asesoría en la elaboración de este trabajo. También agradezco la colaboración y los comentarios del Dr. Jeff Alstott, Mtro. Dylan Fitzpatrick, y Dr. Misha Teplitskiy y de Elizabeth Wiramidjaja.
\\  \href{mailto:carpetri@gmail.com}{\texttt{carpetri@gmail.com}}  \hfill \href{http://detecting-corruption.carlospetricioli.com}{\texttt{detecting-corruption.carlospetricioli.com}} }
\normalsize
\end{center}

\section*{\centering  \normalsize{Resumen} }
% \begin{quotation}
% \scriptsize
\footnotesize
% \noindent 


Para combatir la pobreza mundial, El Banco Mundial presta billones de dólares  para financiar proyectos de desarrollo social. Este trabajo pretende ayudar a los investigadores del Banco Mundial a buscar e identificar patrones que indiquen posible corrupción, colusión y fraude en los datos de los contratos que otorga por medio del uso de modelos por contrato que estiman el riesgo que tienen de ser este tipo de casos. Este trabajo tiene potencial para aportar en la capacidad que tiene el Banco Mundial para detectar este tipo de ofensas, al menos priorizando las investigaciones futuras.

El problema es que  los proveedores que ofrecen bienes y servicios en los proyectos suelen estar seleccionados por medio de un proceso  de subastas/licitaciones que en la práctica deberían ser abiertas y competitivas pero que en la realidad no lo son del todo. Los contratistas manipulan el proceso competitivo coludiéndose con otros contratistas y ofreciendo sobornos a miembros del gobierno, entre otras ofensas. Este tipo de ofensas tienen consecuencias importantes en los precios y en la calidad de entrega de los contratos y proyectos. En este trabajo se desarrollan modelos por contrato para identificar el riesgo que presentan de tener este tipo de ofensas con base en datos históricos de contratos otorgados en los últimos 20 años y en datos de investigaciones que ha realizado el Banco Mundial anteriormente incluyendo empresas y proyectos que se investigaron. 

Para esto, se creó una aplicación que les permite a los investigadores dentro del Banco Mundial, seguir el desarrollo de una empresa en específico a lo largo de distintos países, sectores y en el tiempo. Por medio de esta herramienta, los investigadores del Banco Mundial pueden tener mejor control sobre los contratos que cada empresa ha recibido, así como los distintos nombres con los cuales se ha registrado en las distintas bases de datos y todo además con  un indicador del nivel de riesgo para cada uno de los contratistas del Banco Mundial con base en variables que incluyen, entre otras, datos de la red de empresas/proyectos. La aplicación también incluye visualizaciones de lo anterior.

En conclusión, los datos con los que cuenta actualmente el Banco Mundial son suficientes para identificar el riesgo de corrupción, colusión y fraude y permite que el Banco Mundial investigue de una manera proactiva para determinar qué empresas, proyectos o contratos analizar.


\thispagestyle{plain}
\vfill
\scriptsize \noindent \textbf{Palabras clave:} Corrupción, Colusión, Fraude, Banco Mundial.
\normalsize