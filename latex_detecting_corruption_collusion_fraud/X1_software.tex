\chapter{Software}
\label{chap_software}

\noindent The software in used in this project is all free software. All the code was written using the  language \texttt{R Project for Statistical Computing}\footnote{``R is available as Free Software under the terms of the Free Software Foundation’s GNU General Public License in source code form. It compiles and runs on a wide variety of UNIX platforms and similar systems (including FreeBSD and Linux), Windows and MacOS.'' See \cite{r_pagina}.
} \footnote{See \cite{R:Bloomfield:2014} for an R user guide.}, and \textit{Python}\footnote{``Python is developed under an OSI-approved open source license, making it freely usable and distributable, even for commercial use. Python's license is administered by the Python Software Foundation.'' See \cite{python_about}.}. To summarize and visualize the data, all the graphs were made with \texttt{ggplot2} \parencite{wickham_ggplot} a package for R writen by Hadley Wickham. To create the global map using the spatial data, the project uses \texttt{Quantum-Gis (QGIS)}\footnote{See \cite{clifford1981} for details of what's spatial data.} \footnote{GIS refers to Geographical Information Systems and it refers to the set of techniques that uses spatial data, see \cite{GIS05}  for more details.}. To make the searches at Google using many Amazon Web Service's computers it uses \texttt{Parallel} \parencite{Tange2011a} to distribute the tasks. For the development, edition an writing, this project uses \LaTeX - A document preparation system \cite{latex}.