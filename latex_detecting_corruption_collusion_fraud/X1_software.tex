% !TEX root = A0_thesis_detecting_collusion_corruption_fraud.tex
\chapter{Software}
\label{chap_software}


\noindent The software in used in this project is free software. All the code was written using the  language \texttt{R Project for Statistical Computing}\begin{quote}``R is available as Free Software under the terms of the Free Software Foundation's GNU General Public License in source code form. It compiles and runs on a wide variety of UNIX platforms and similar systems (including FreeBSD and Linux), Windows and MacOS.'' \footnote{See \cite{R:Bloomfield:2014} for an R user guide.} \cite{r_pagina},\end{quote}
and \textit{Python}
\begin{quote}``Python is developed under an OSI-approved open source license, making it freely usable and distributable, even for commercial use. Python's license is administered by the Python Software Foundation.''\cite{python_about}.\end{quote}

\texttt{Python} is general-purpose programming language and \texttt{R} is  a programming environment optimized for statistics. They're all widely used in the business and academic worlds, and the modules help users working with those languages to connect to the World Bank Development Indicators API and access the latest data.

To summarize and visualize the data, all the graphs were made with \texttt{ggplot2} \parencite{wickham_ggplot} a package for \texttt{R} written by Hadley Wickham. To create the global map using the spatial data, the project uses \texttt{Quantum-Gis (QGIS)}\footnote{See \cite{clifford1981} for details of what's spatial data.} \footnote{GIS refers to Geographical Information Systems and it refers to the set of techniques that uses spatial data, see \cite{GIS05}  for more details.}. To make the searches at Google using many Amazon Web Service's computers it uses \texttt{Parallel} \parencite{Tange2011a} to distribute the tasks. For the development, edition an writing, this project uses \LaTeX - A document preparation system \cite{latex}.

\texttt{Python} \texttt{pandas} is an open source, BSD-licensed library providing high-performance, easy-to-use data structures and data analysis tools for the Python programming language.

\section{Amazon Web Service}

Amazon Web Services (AWS) provides computing resources and services that you can use to build applications within minutes at pay-as-you-go pricing. For example, you can rent a server on AWS that you can connect to, configure, secure, and run just as you would a physical server. The difference is the virtual server runs on top of a planet-scale network managed by AWS. The most common service is the Amazon Elastic Compute Cloud (Amazon EC2) machine. EC2 is a web service that provides resizable compute capacity in the cloud. It is designed to make web scale cloud computing easier for developers. Amazon EC2’s simple web service interface allows you to obtain and configure capacity with minimal friction. It provides you with complete control of your computing resources and lets you run on Amazon's proven computing environment. Amazon EC2 reduces the time required to obtain and boot new server instances to minutes, allowing you to quickly scale capacity, both up and down, as your computing requirements change. Amazon EC2 changes the economics of computing by allowing you to pay only for capacity that you actually use. Amazon EC2 provides developers the tools to build failure resilient applications and isolate themselves from common failure scenarios.

The simplest way to connect to an EC2 is by using \texttt{ssh} command line tools. You'll specify the private key (.pem) file and user\_name@public\_dns\_name. 
\begin{lstlisting}[language=bash]
chmod 400 /path/my-key-pair.pem
ssh -i /path/my-key-pair.pem \
ec2-user@ec2-198-51-100-1.compute-1.amazonaws.com
\end{lstlisting}

To transfer a file to your instance using the instance's public DNS name. For example, if the name of the private key file is my-key-pair, the file to transfer is	\texttt{SampleFile.txt}, and the public DNS name of the instance is \begin{lstlisting}
ec2-198-51-100-1.compute-1.amazonaws.com
\end{lstlisting} use the following command to copy the file to the \texttt{ec2-user} home directory.
\begin{lstlisting}
scp -i /path/my-key-pair.pem SampleFile.txt \
ec2-user@ec2-198-51-100-1.compute-1.amazonaws
\end{lstlisting}
