\chapter{The World Bank's Data}\label{chap_data}

As shown on chapter \ref{chap_intro}, collecting the necessary data for any Data Science project is almost always a huge task because the reality tends to be that some of the most relevant variables commonly go unmeasured or are private. This project uses  two types of data, private and public. This chapter describes the two main types of data used for this project.
Section \ref{sec_inv_data} describes the characteristics of private data and the scope it has. It also explains about the confidentiality agreement Integrity Vice Presidency (INT) offered this project in exchange for investigations data. Section \ref{sec_public_data} explains how to download public data from the Wold Bank's servers using coding tools.

\section{Investigations data}\label{sec_inv_data}

The most valuable data for this project was the private data which the partners at INT made available for the project while maintaining a spirit of openness. A confidentiality agreement was signed in order to give access to the investigations database. The only condition was that all the data and analyses had to be stored and processed on a remote server with an encryption and security measurements that satisfied the World Bank's standards. 

In this case the server was an Amazon Web Service's (AWS), Amazon Elastic Compute Cloud (Amazon EC2) machine. EC2 is a web service that provides resizable compute capacity in the cloud. It is designed to make web scale cloud computing easier for developers\footnote{``Amazon EC2’s simple web service interface allows you to obtain and configure capacity with minimal friction. It provides you with complete control of your computing resources and lets you run on Amazon’s proven computing environment. Amazon EC2 reduces the time required to obtain and boot new server instances to minutes, allowing you to quickly scale capacity, both up and down, as your computing requirements change. Amazon EC2 changes the economics of computing by allowing you to pay only for capacity that you actually use. Amazon EC2 provides developers the tools to build failure resilient applications and isolate themselves from common failure scenarios.'' \parencite{aws_es2}}. For more details on how to setup an AWS computer see \parencite{aws_start} and appendix \ref{chap_software}. In other words, the data was stored in an AWS machine on the cloud that had an encrypted folder with the specific files. Since all the data had to stay on the cloud, all the processing had to be done on the cloud too. For this purpose, 


Disambiguation

\section{Public data}\label{sec_public_data}

Fortunately for the project, not all the data was private. This work uses the Open Data\footnote{Go to \cite{wb_data}.}  and the Data Bank websites from the World Bank, which offer free access to comprehensive, downloadable indicators about development in countries around the globe. A good practice in Data Science is to generate code so that all results can be easily replicated. That was the reason why, for this project all the data that the World Bank bank provides publicly by an Application Programming Interface (API) was collected with reliable code. 

Thanks to prior World Bank's work, accessing the World Bank Data APIs with code in languages such as \texttt{Python}, \texttt{R}, \texttt{Ruby} and \texttt{Stata} is a simple task. The World Bank has a blog where they explain in detail how to use the APIs.See the \cite{wb_api}, \cite{wb_python} and \cite{wb_r} for more details on how to do this. To put things in context, \texttt{Python} and \texttt{Ruby} are general-purpose programming languages, and \texttt{Stata} and \texttt{R} are programming environments optimized for statistics. They're all widely used in the business and academic worlds. The World Bank generates modules to those languages which help users to connect to the World Bank Development Indicators API and access the latest data.

For example, in \texttt{python}, the \texttt{wbdata} module by Oliver Sherouse offers easy access to all the data in the World Bank's APIs. It also plays nicely with Wes McKinney’s  \texttt{pandas} analysis library\footnote{See appendix \ref{chap_software} for \texttt{pandas} details.}. \texttt{Wbdata} is a simple python interface to find and request information from the World Bank's various databases, either as a dictionary containing full meta data or as a \texttt{pandas} Data Frame. Currently, \texttt{wbdata} wraps most of the World Bank API, and also adds some convenience functions for searching and retrieving information.

In \texttt{R}, the \texttt{WDI} module by Vincent Arel-Bundock offers convenient access to the data in the World Bank's API. For fast searching, the \texttt{WDI} package ships with a local list of available data series. This local list can be updated to the latest version using the \texttt{WDIcache} function \parencite{wb_r}. Similar tools are available to languages such as Ruby and Stata.

\subsection{World Development Indicators}

The World Bank has a major data base called: World Development Indicators (WDI). WDI is the primary World Bank collection of development indicators, compiled from officially recognized international sources. This database represents the most current and accurate global development data available, and includes national, regional and global estimates.

Being this big, and according to purpose of this project, only a few indicators were selected. The next list shows the ones that were considered. This list includes indicators related to 

\begin{enumerate}[1.]
\item \texttt{IC.BUS.DISC.XQ}	Private Sector \& Trade: Business environment. Business extent of disclosure index (0=less disclosure to 10=more disclosure)	Disclosure index measures the extent to which investors are protected through disclosure of ownership and financial information. The index ranges from 0 to 10, with higher values indicating more disclosure.
\item \texttt{IC.FRM.CMPU.ZS}	Private Sector \& Trade: Business environment. Firms competing against unregistered firms (\% of firms).
\item \texttt{IC.FRM.CORR.ZS}	Private Sector \& Trade: Business environment	Informal payments to public officials (\% of firms).
\item \texttt{IC.FRM.INFM.ZS}	Private Sector \& Trade: Business environment	Firms that do not report all sales for tax purposes (\% of firms).
\item \texttt{IC.LGL.CRED.XQ}	Private Sector \& Trade: Business environment	Strength of legal rights index (0=weak to 10=strong).
\item \texttt{IC.LGL.DURS}	Private Sector \& Trade: Business environment	Time required to enforce a contract (days).
\item \texttt{IC.TAX.GIFT.ZS}	Private Sector \& Trade: Business environment	Firms expected to give gifts in meetings with tax officials (\% of firms).
\item \texttt{IQ.CPA.PROP.XQ}	Public Sector: Policy \& institutions	CPIA property rights and rule-based governance rating (1=low to 6=high).
\item \texttt{IQ.CPA.TRAN.XQ}	Public Sector: Policy \& institutions	CPIA transparency, accountability, and corruption in the public sector rating (1=low to 6=high).
\item \texttt{NY.GDP.PCAP.CD}	Economic Policy \& Debt: National accounts: USD at current prices: Aggregate indicators	GDP per capita (current US\$).
\item \texttt{SE.PRM.PRSL.ZS}	Education: Efficiency	Persistence to last grade of primary, total (\% of cohort).
\item \texttt{SI.POV.GINI}	Poverty: Income distribution GINI index.
\item \texttt{SL.UEM.TOTL.NE.ZS}	Social Protection \& Labor: Unemployment	Unemployment, total (\% of total labor force) (national estimate).
\end{enumerate}




