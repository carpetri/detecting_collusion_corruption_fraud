% !TEX root = A0_thesis_detecting_collusion_corruption_fraud.tex
\chapter{Introduction}\label{chap_intro}

The World Bank Group lends billions of dollars each year to fund development projects in its efforts to reduce global poverty. This project helps investigators at the Bank search for patterns of collusion, corruption, and fraud in its contracts data, using models of contract-specific risk. By developing an automated approach to detect these offenses this project can help the World Bank efficiently target future investigations.

Several things are required for a project of this type to achieve success. The following list outlines the requirements for a successful project, some requirements easier to satisfy so it is ordered from easiest to most difficult. These alone do not guarantee success, but success is nearly impossible without them \parencite{dssg_proj_list}.   \textit{1. A solvable problem}. This project cannot solve poverty, but it can help alleviate it by reducing corruption, collusion and fraud. 
\textit{2. A challenging problem}. Challenging problems encourage teamwork, spawn creative solutions, and play a key role in a Data Scientist's ability to solve real-world problems and an understanding, excitement, and passion for solving problems with social impact. For example, this World Bank project involved  search-engine expertise to find links between corrupt applicants online. 
\textit{3. An important problem with social impact}. Working in a big project is an investment, not only financially, but also opportunistically (when someone choose to do a project, someone is choosing not to do another). It is important to dedicate the limited resources to substantial problems. Each project must meet an operational need for the partner organization and must have a tangible connection to ``social good.'' For example, this project helps more people over fewer people and that solve chronic problems over temporary problems. 
\textit{4. A motivated, capable, and committed partner}. No project can succeed without a fully invested project partner. Project partners understand the problem, they have subject-matter expertise, and they ultimately decide how a Data Scientist's work is used. Partners often look at the problem differently, which is important for solving tough problems.  Partners provide insight into the problem and to guide Data Scientist to develop a solution. This demands a lot from partners. It often requires partners stretching themselves and asking themselves hard questions. It also requires time. For this project, the World Bank helped scope the problem before it started, they gave a complete presentation about their work, they were available for at least once a week for feedback and discussion during the duration of the project, and they are actually using the results of this work when it finished. 
\textit{5. Appropriate, relevant data}. Getting the data a project  needs is almost always the biggest challenge. Important things go unmeasured or unrecorded or, more commonly, cannot be shared. Many projects involve  sensitive information. Getting lawyers to agree on data and code sharing can take months. It is important for Data Scientist to be flexible. Partners have anonymized data (while keeping it useful at an individual level), conducted background checks, and require Data Scientist to do analyses on their internal computer systems (remotely). All this while maintaining a spirit of openness.  In this case, the World Bank contributed with all the relevant data they have in order to build a solution that's appropriate, effective, and easily deployed.

\section{The World Bank}\label{sec_intro_wb}

The World Bank was created in 1944 as one institution and has evolved into an association of five development institutions. It is composed by  the International Bank for Reconstruction and Development (IBRD), the International Development Association (IDA), the International Finance Corporation (IFC), the Multilateral Guarantee Agency (MIGA), and the International Center for the Settlement of Investment Disputes (ICSID). In its interior it was formed by a large group of engineers and financial analysts who work from Washington, D.C. Now, they have a very heterogeneous and distinct staff that includes economist, public policy experts, sector experts and social scientists and about one third of the workers has spread to the country offices. They have more than 10,000 employees in more than 120 offices worldwide. While reconstruction remains an important part of their objectives, however, at today's World Bank, poverty reduction through an inclusive and sustainable glottalization remains the overarching goal. For details see \parencite{wb_history}. 

The World Bank Group has set two goals for the world to achieve by 2030; first, end extreme poverty by decreasing the percentage of people living on less than \$1.90 a day to no more than 3\%; second, promote shared prosperity by fostering the income growth of the bottom 40\% for every country; third, the World Bank is a vital source of financial and technical assistance to developing countries around the world. The World Bank Group is not a bank in the ordinary sense but a unique partnership to reduce poverty and support development \parencite{wb_about}.

The World Bank Group provides Financial Products and Services by giving out low-interest loans, zero to low-interest credits, and grants to developing countries.
\begin{quote} ``These support a wide array of investments in such areas as education, health, public administration, infrastructure, financial and private sector development, agriculture, and environmental and natural resource management. Some of our projects are cofinanced with governments, other multilateral institutions, commercial banks, export credit agencies, and private sector investors.''\parencite{wb_about}. \end{quote}They offer financing through trust fund partnerships with bilateral and multilateral donors. For example, many partners have asked the Bank to help manage initiatives that address needs across a wide range of sectors and developing regions. Projects vary widely in scale and scope, ranging from developing hydropower systems\footnote{ WASHINGTON, March 20, 2014–Sub-Saharan Africa is blessed with large hydropower resources that can bring electricity to homes, power businesses and industry, light clinics and schools, and spur economic activity, creating jobs and improving human well-being.  Yet, only 10\% of this hydropower potential has been mobilized, weakening the fight to end poverty and boost shared prosperity on the continent. See \cite{wb_hydro}} to rehabilitating coral reefs\footnote{ Wakatobi, Indonesia, June 5, 2014 - As the world's largest archipelago, Indonesia is blessed with at least 5.1 million hectares of coral reefs. However, almost 65 percent of the reefs are now considered threatened from overfishing. Almost half are considered threatened specifically from destructive fishing practices. Nadjib Prasyad runs the Fisheries Office in Wakatobi, South-east Sulawesi, and he laments the various activities that destroy the reefs and consequently threaten the livelihood of the villages: fish bombing, sand extraction, collection of the reefs themselves. Prasyad says that, once the reefs die, so do the fish: “We have nothing except our coral reefs.So we have to really protect them since they're the only source of our region's development. See \cite{wb_coral}} to improving roads, health, education and agriculture systems\footnote{ ``We didn't use to have bus service to the communities; now we have it every hour.'' ``Now we have a paved highway and signs; it's comfortable for traveling.'' ``When there was a drought, there was no water for the animals, for the pasture, for irrigating the produce we consumed and much less for crops for sale.'' ``With the water, people returned to the community and it is improving their quality of life.'' Today, these are some of the phrases that can be heard on the roads of Chimborazo, one of the largest provinces in Ecuador’s central highlands. See \cite{wb_roads} }.

This work focuses in these Financial Products and Services. It analyzes their contracts (see the chapter \ref{chap_procurements} for more details) by using historical data on over 300,000 major contracts funded by World Bank in the form of low-interest loans, zero to low-interest credits, and grants to developing countries covering  from the past 20 years. Historical data including such features as company name, country, sector, and total award amount.

The World Bank also provides Innovative Knowledge Sharing, by offering support to developing countries through policy advice, research and analysis, and technical assistance. \begin{quote}
``The analytical work often underpins World Bank financing and helps inform developing countries' own investments. In addition, we support capacity development in the countries we serve. We also sponsor, host, or participate in many conferences and forums on issues of development, often in collaboration with partners''\parencite{wb_about}.
\end{quote}

This is why this project becomes of high value for the World Bank Group. On one hand, this work helps them to provide much better Financial Products and Services by reducing the cost that corruption, collusion and fraud generates in the process of lending money to developing countries throughout the world's developing countries. On the other, this project contribute to their Innovative Knowledge Sharing by supplying them with the necessary tools and algorithms they need in order to share corruption, collusion and fraud analysis' results.

This project relies on the fact that the World Bank group, in order to ensure that countries can access the best global expertise and help generate cutting-edge knowledge, is constantly seeking to improve the way it shares its knowledge and engages with clients and the public at large. This includes  measurable results, improving every aspect of their work like how projects are designed, how information is made available, and how to bring our operations closer to client governments and communities and also includes  their Open Development, a set of free, easy-to-access tools, research and knowledge to help people address the world's development challenges. This work would not be possible without all this previous work and is a virtuous circle because it contributes by helping them achieve this goal.

This work uses the Open Data\footnote{Go to \cite{wb_data}} website from the World Bank, which offers free access to comprehensive, downloadable indicators about development in countries around the globe, data needed in order to detect and predict patterns of collusion, corruption, and fraud in its contracts data by using models of contract-specific risk. 

On chapter \ref{chap_procurements} BLA BLA BLA


On chapter \ref{chap_procurements}




